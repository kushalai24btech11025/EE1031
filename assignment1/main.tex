%\iffalse
\let\negmedspace\undefined
\let\negthickspace\undefined
\documentclass[journal,12pt,twocolumn]{IEEEtran}
\usepackage{cite}
\usepackage{circuitikz}
\usepackage{amsmath,amssymb,amsfonts,amsthm}
\usepackage{algorithmic}
\usepackage{graphicx}
\usepackage{textcomp}
\usepackage{xcolor}
\usepackage{txfonts}
\usepackage{listings}
\usepackage{enumitem}
\usepackage{mathtools}
\usepackage{gensymb}
\usepackage{comment}
\usepackage[breaklinks=true]{hyperref}
\usepackage{tkz-euclide} 
\usepackage{listings}
\usepackage{gvv}                                        
\def\inputGnumericTable{}                                 
\usepackage[latin1]{inputenc}                                
\usepackage{color}                                            
\newtheorem{theorem}{Theorem}[section]
\usepackage{array}                                            
\usepackage{longtable}                                       
\usepackage{calc}                                             
\usepackage{multirow} 
\usepackage{multicol}
\usepackage{hhline}                                           
\usepackage{ifthen}                                           
\usepackage{lscape}
\usepackage{multicol}
\newtheorem{problem}{Problem}
\newtheorem{proposition}{Proposition}[section]
\newtheorem{lemma}{Lemma}[section]
\newtheorem{corollary}[theorem]{Corollary}
\newtheorem{example}{Example}[section]
\newtheorem{definition}[problem]{Definition}
\newcommand{\BEQA}{\begin{eqnarray}}
\newcommand{\EEQA}{\end{eqnarray}}
\newcommand{\define}{\stackrel{\triangle}{=}}
\theoremstyle{remark}
\newtheorem{rem}{Remark}
\begin{document}
\bibliographystyle{IEEEtran}
\vspace{3cm}
\newpage
\bigskip
\renewcommand{\thefigure}{\theenumi}
\renewcommand{\thetable}{\theenumi}

\title{Quadratic equations and Inequations(Inequalities) }
\author{ai24btech11025 KUSHAL% <-this % stops a space
}
\maketitle\Large{SectionC.MCQs with One Correct Answer}
\begin{enumerate}[start=31]
\item If one root is square of the other root of the equation $x^{2}+px+q=0$, then the relationship between p and q is \hfill (2004S)

\begin{enumerate}
    \item $p^{3}-q\brak{3p-1}+q^{2}=0$
    \item $p^{3}-q\brak{3p+1}+q^{2}=0$
    \item $p^{3}+q\brak{3p-1}+q^{2}=0$
    \item $p^{3}+q\brak{3p+1}+q^{2}=0$
\end{enumerate}
\item Let a,b,c be the sides of the triangle where $a\neq b\neq c$ and $\lambda  R$. If the roots of the equation 

$x^{2}+2\brak{a+b+c}x+3\lambda\brak{ab+bc+ca}=0$ are real, then \hfill(2006-3M,-1)
\begin{multicols}{2}
    \begin{enumerate}
        \item $\lambda<\frac{4}{3}$
        \item $\lambda>\frac{5}{3}$
        \item $\lambda \brak{\frac{1}{3},\frac{5}{3}}$
        \item $\lambda \brak{\frac{1}{3},\frac{4}{3}}$
    \end{enumerate}
\end{multicols}
\item Let $\alpha,\beta$ be the roots of the equation $x^{2}-px+r=0$ and $\frac{\alpha}{2},2\beta$ be the roots of the equation $x^{2}-qx+r=0$. Then the value of r is \hfill (2007-1marks)

\begin{enumerate}
    \item $\frac{2}{9}\brak{p-q}\brak{2q-p}$
    \item $\frac{2}{9}\brak{q-p}\brak{2p-q}$
    \item $\frac{2}{9}\brak{q-2p}\brak{2q-p}$
    \item $\frac{2}{9}\brak{2p-q}\brak{2q-p}$
\end{enumerate}
\item let p and q be real numbers such that $p\neq 0,p^{3}\neq q and p^{3}\neq -q.$ If $\alpha and \beta$ are nonzero complex numbers satisfying $\alpha+\beta=-p and \alpha^{3}+\beta{3}=q$,then a quadratic equation having $\frac{\alpha}{\beta} and \frac{\beta}{\alpha}$ as its roots 

is \hfill (2010)
\begin{enumerate}
    \item $\brak{p^{3}+q}x^{2}-\brak{p^{3}+2q}x+\brak{p^{3}+q}=0$
    \item $\brak{p^{3}+q}x^{2}-\brak{p^{3}-2q}x+\brak{p^{3}+q}=0$
    \item $\brak{p^{3}+q}x^{2}-\brak{5p^{3}-2q}x+\brak{p^{3}-q}=0$
    \item $\brak{p^{3}+q}x^{2}-\brak{5p^{3}+2q}x+\brak{p^{3}-q}=0$
\end{enumerate}
\item Let $\brak{x_0,y_0}$ be the solution of the following equatoions

$\brak{2x}^{ln2}=\brak{3y}^{ln3}$

$3^{lnx}=2^{lny}$

then $x_{0}$ is \hfill (2011)
\begin{multicols}{4}
\begin{enumerate}
    \item $\frac{1}{6}$
    \item $\frac{1}{3}$
    \item $\frac{1}{2}$
    \item 6
\end{enumerate}
    
\end{multicols}
\item Let $\alpha and \beta$ be the roots of $x^{2}-6x-2=0$, with $\alpha>\beta$. if $a_{n}=\alpha^{n}-\beta{n}$ for $n\geq 1$,then the value of $\frac{a_{10}-2a_{8}}{2a_{9}}$ is \hfill (2011)
\begin{multicols}{4}
\begin{enumerate}
    \item 1
    \item 2
    \item 3
    \item 4    
\end{enumerate}
    
\end{multicols}
\item A value of b for which the equations 

$x^{2}+bx-1=0$

$x6{2}+x+b=0$

have one root in common is \hfill (2011)
\begin{multicols}{2}
\begin{enumerate}
    \item $-\sqrt{2}$
    \item $-i\sqrt{3}$
    \item $i\sqrt{5}$
    \item $\sqrt{2}$
\end{enumerate}
    
\end{multicols}
\item The quadratic equation $p\brak{x}=0$ with real coefficients has purely imaginary roots. Then the equation$p\brak{p\brak{x}}=0$ has \hfill (JEE Adv,2014)
\begin{enumerate}
    \item one purely imaginary root
    \item all real roots
    \item two real roots and two purely imaginary roots 
    \item neither real nor imaginary roots
\end{enumerate}
\item let $-\frac{\pi}{6}<\theta<-\frac{\pi}{12}$. Suppose $\alpha_{1}and\beta_{1}$ are the roots of the equation $x^{2}-2x\sec \alpha+1=0$ and $\alpha_{2}and\beta_{2}$ are the roots of the equation $x^{2}-2x\tan \theta-1=0$. If $\alpha_{1}>\beta_{1}$ and $\alpha_{2}>\beta_{2}$,then $\alpha_{1}+\beta_{2}$ equals \hfill (JEE Adv.2016)
\begin{enumerate}
    \item $2\brak{\sec\theta-\tan\theta}$
    \item $2\sec\theta$
    \item $-2\tan\theta$
    \item 0
    
\end{enumerate}
\end{enumerate}
\maketitle\Large{Section D.MCQs with One or More than One Correct}
\begin{enumerate}
\item for real x, the function $\frac{\brak{x-a}\brak{x-b}}{x-c}$ will assume all real values provided \hfill (1984-3Marks)
\begin{multicols}{2}
\begin{enumerate}
    \item $a>b>c$
    \item $a<b<c$
    \item $a>c>b$
    \item $a<c<b$
\end{enumerate}
    
\end{multicols}
\item if S is the set of all real x such that $\frac{2x-1}{2x^{3}+3x^{2}+x}$ is positive,then S contains \hfill (1996-2Marks)
\begin{multicols}{2}
\begin{enumerate}
    \item \brak{-infinity,-\frac{3}{2}}
    \item \brak{-\frac{3}{2},-\frac{1}{4}}
    \item \brak{-\frac{1}{4},\frac{1}{2}}
    \item \brak{\frac{1}{2},3}
\end{enumerate}
    
\end{multicols}
\item if a,b and c are distinct positive numbers, then the expression $\brak{b+c-a}\brak{c+a-b}\brak{a+b-c}-abc$ is \hfill (1986-2 Marks)
\begin{enumerate}
    \item positive
    \item negative
    \item non-positive
    \item non-negative
    \item none of these
\end{enumerate}
\item if a,b,c,d and p are distinct real numbers such that $\brak{a^{2}+b^{2}+c^{2}}p^{2}-2\brak{ab+bc+cd}p+\brak{c^{2}}+c^{2}+d^{2}\leq 0$ then a,b,c,d \hfill (1987-2 Marks)
\begin{enumerate}
    \item are in A.P.
    \item are in G.P.
    \item are in H.P.
    \item satisfy $ab=cd$
    \item satisfy none of these
\end{enumerate}
\item The equation

$x^{3/4\brak{\log_{2} x}^{2}+\log_{2} x-5/4}=\sqrt{2}$

has \hfill (1989-2 Marks)
\begin{enumerate}
    \item at least one real solution
    \item exactly three solutions
    \item exactly one irrational solution
    \item complex roots
\end{enumerate}
\item The product of n positive numbers is unity Then their sum is \hfill (1991-2 Marks)
\begin{enumerate}
    \item a positive integer
    \item divisible by n
    \item equal to $n+\frac{1}{n}$
    \item never less than n
\end{enumerate}


\end{enumerate}


\end{document}